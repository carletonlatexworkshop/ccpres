% Template for Carleton student presentations
% Author: Andrew Gainer-Dewar, 2013
\documentclass{beamer}

% Beamer has facilities to format handouts and notes to go along with your slides,
% but we won't use them here.
\mode<presentation>

% Beamer comes with *lots* of themes. Boadilla is a nice choice;
% it's elegant and stays out of the way.
% Many of the standard themes are very busy, with tables of contents in the
% sidebar and other messiness. This is usually just a distraction for your audience.
% (If they're looking at the ToC, it's because they're bored!)
\usetheme{Boadilla}

% We then load in the custom Carleton color scheme
\usecolortheme{carl}

% The lxfonts package is a nice choice for a presentation.
% lmodern and arev are also good choices (but the latter looks a
% bit clunky with math).
\usepackage[T1]{fontenc}
\usepackage{lxfonts}

% By default, Beamer puts some useless cruft at the bottom of the page.
% We turn it off.
\setbeamertemplate{navigation symbols}{}

% Set your title and author data here.
% The optional arguments will be used in the running footer;
% simply remove them if you want to re-use the long versions.
\title[Example presentation]{An example of a presentation with \LaTeX{} and Beamer}
\subtitle[Test]{Test subtitle}
\author[Jeremiah \and Langstrom]{Lori T.~Jeremiah \and Ray B.~Langstrom}
\institute[Carleton]{Carleton College}
\date{February 21, 2014}

\begin{document}

% Each slide goes in a frame environment.
% The first one should be your title page, which is
% set up automatically by Beamer.
\begin{frame}
  \titlepage

  \begin{center}
    Project advisor: M.~Holmes \\
    Dept.~of Mathemagical Inquiry
  \end{center}
\end{frame}

% Some people like to include a table of contents.
% (I find them tacky; your audience shouldn't need a map!)
\begin{frame}
  \frametitle{Contents}
  \tableofcontents
\end{frame}

\section{Basics}
% Every frame should have a \frametitle
\begin{frame}
  \frametitle{Title of the first frame}
  This frame has very little content.

  \begin{alertblock}{Caution}
    This example is meant to be used alongside its source!
    If you are only viewing the PDF, this document will not be useful to you.
  \end{alertblock}
\end{frame}

\begin{frame}[fragile] % The argument "fragile" protects \verb, which is badly-behaved.
  \frametitle{Text formatting}
  Beamer supports the same text-formatting commands as ordinary \LaTeX{}.

  However, for best results, you should use \verb|\alert| instead of \verb|\emph|:

  \begin{itemize}
  \item \emph{emph text}
  \item \alert{alert text}
  \end{itemize}

  \textbf{Bold text} should be avoided; you can't be sure it will be distinct on a projector.
\end{frame}

\section{Blocks}
\begin{frame}
  \frametitle{Use of block environments I}
  Beamer supports the standard block environments.

  \begin{theorem}
    You can write theorems.
  \end{theorem}

  \begin{proof}
    They can have proofs!
  \end{proof}
\end{frame}

\begin{frame}
  \frametitle{Use of block environments II}
  \begin{example}
    You can also give examples.
  \end{example}

  \begin{definition}
    And define \alert{terms}.
  \end{definition}
\end{frame}

\begin{frame}
  \frametitle{Use of block environments III}
  \begin{block}{Custom block}
    You can even create custom blocks!
  \end{block}

  \begin{exampleblock}{Custom example}
    And custom example blocks!
  \end{exampleblock}
\end{frame}

\section{Math}
\begin{frame}
  \frametitle{Using math in Beamer}

  Math works both inline: $x^{2} + y^{2} = z^{2}$
  and display:
  \[ \int_{-\infty}^{\infty} e^{-x^{2}} \, \mathrm{d}x = \sqrt{\pi} \]
\end{frame}

\section{Lists}
\begin{frame}
  \frametitle{Lists in Beamer}
  Beamer also has support for ordered and unordered lists.

  \begin{enumerate}
  \item An ordered item.
  \item Another ordered item.
    \begin{itemize}
    \item An unordered subitem.
    \item Another unordered subitem.
    \end{itemize}
  \item A third ordered item.
  \end{enumerate}
\end{frame}

\section{Overlays}
\begin{frame}
  \frametitle{Overlay specifications}
  Beamer allows you to set up animated slides using ``overlays''.
  Use this with caution---less is more!

  Overlays are created by specifying the numbers of the subslides on which the object should appear.
\end{frame}

\begin{frame}[fragile] % The "fragile" argument protects \verb, which is badly-behaved
  \frametitle{Example of overlay with itemize}
  The following items will appear one at a time.
  Each is created using the \verb|\item| call shown.
  \begin{itemize}
  \item<1-> \verb|\item<1->|
  \item<2-> \verb|\item<2->|
  \item<3-> \verb|\item<3->|
  \item<4-> \verb|\item<4->|
  \end{itemize}
\end{frame}

\subsection{Simplified syntax}
\begin{frame}[fragile]
  \frametitle{Example of simplified overlay with itemize}

  Beamer supports a simplified syntax for revealing lists one item at a time.
  \begin{block}{Code}
  \begin{verbatim}
\begin{itemize}[<+->]
  \item First item
  \item Second item
  \item Third item
\end{itemize}
  \end{verbatim}
  \end{block}

  \begin{itemize}[<+->]
    \item First item
    \item Second item
    \item Third item
  \end{itemize}
\end{frame}

\begin{frame}[fragile]
  \frametitle{Example of simplified overlay with text}
  You can also use a simplified syntax to reveal a slide from top to bottom using \verb|\pause|.

  \pause

  This text will appear only on the second slide.
\end{frame}

\section{Structure and ToC}
\begin{frame}[fragile]
  \frametitle{Document structure}
  A Beamer presentation can be structured using \verb|\section| and \verb|\subsection|, just like a paper.

  These commands go \alert{between} the slides.
  They do not change the text on your slides, but they show up in the Table of Contents, and some themes show them in a sidebar or tree.
\end{frame}

\begin{frame}
  \frametitle{Tables of contents}
  Beamer supports automatically building tables of contents.

  \begin{alertblock}{Caution}
    Be careful how you use the table of contents.
    Just throwing an outline slide up and then using a minute to read it out is a waste of your audience's time.
  \end{alertblock}
\end{frame}

\section{Slide formatting}
\begin{frame}
  \frametitle{Multicolumn slides}
  Beamer supports slides with multiple columns.

  \begin{columns}
    \column{0.4\textwidth} %Using .4\textwidth leaves some gutter space between the columns, which is good with blocks.
    \begin{theorem}
      Here is a theorem.
    \end{theorem}

    \column{0.4\textwidth}
    \begin{proof}
      And here is its proof.
    \end{proof}
  \end{columns}
\end{frame}
\end{document}